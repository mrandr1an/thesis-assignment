\subsection{Γράφοντας το Project}

Θα δούμε πως η υλοποίηση του Project με βάση το specification που
έχουμε θέσει είναι ιδιαίτερα απλή. Παρόλα αυτά δεν προσφέρονται βιβλιοθήκες
για τον HC-SR04 όπως στην περίπτωση του ESP-IDF. Παράλληλα το API της
βιβλιοθήκης του MQTT είναι πολύ δύσκολα διαχειρήσημο οπότε θα
χρειαστεί κάποιου είδους βελτίωση χωρίς απαραίτητα να γράψουμε το
πρωτόκολλο από την αρχή (το οποίο όμως φαίνεται να είναι η ιδανική
λύση). Με βάση την γνώση που έχουμε δώσει στην εισαγωγή αυτού του
κεφαλαίου όμως γίνεται ξεκάθαρο πως αυτό δεν αποτελεί πρόβλημα καθώς η
εκφραστηκότητα της Rust κάνει το γράψιμο κώδικα να φαίνεται σαν πολύ
οργανική διαδικασία.

Μια επιλογή για να ξεκινήσουμε είναι να δημιουργήσουμε ένα Cargo.toml και
να θέσουμε τις βιβλιοθήκες που θέλουμε να μεταγλωττίσουμε μαζί με το Project.
Όμως υπάρχει μια καλύτερη λύση, το \verb|esp-generate --chip esp32c6 my_project|
δημιουργεί τα απαραίτητα αρχεία με τις βιβλιοθήκες που είναι απαραίτητες ανάλογα με
τις ρυθμίσεις μας. Προφανώς αν θέλουμε προσθέτουμε κιάλλες μπορούμε είτε από ένα τερματικό
\verb|cargo add library| είτε με το συντακτικό TOML που έχουμε δείξει παραπάνω.
Το τελικό directory έχει της εξής δομή:

\begin{figure}[htbp]
  \centering
\begin{forest}
for tree={
  font=\ttfamily,
  grow'=0,
  child anchor=west,
  parent anchor=south,
  anchor=west,
  calign=first,
  edge path={
    \noexpand\path [draw, \forestoption{edge}]
      (!u.south west) +(5pt,0) |- (.child anchor)\forestoption{edge label};
  },
  before typesetting nodes={
    if n=1
      {insert before={[,phantom]}}
      {}
  },
  fit=band,
  before computing xy={l=15pt},
}
[.
  [build.rs]
  [.cargo
    [config.toml]
  ]
  [Cargo.toml]
  [.gitignore]
  [rust-toolchain.toml]
  [src
    [bin
      [main.rs]
    ]
    [lib.rs]
  ]
]
\end{forest}
\caption{Project folder structure.}
\label{fig:project_structure}
\end{figure}

Το project θα αποτελείτε από δύο δικές μας βιβλιοθήκες που θα είναι υπεύθυνες για:

\begin{enumerate}
  \item Την λειτουργία του αισθητήρα απόστασης.
  \item Την λειτουργεία του πρωτόκολλου MQTT.
\end{enumerate}

\subsubsection{HC-SR04 Driver}

Ο σένσορας για να λειτουργήσει χρειάζεται εκτός από το VCC και το ground pin
ένα input και ένα output. Οπότε γράφουμε σε ένα αρχείο \verb|hcsr04.rs|:

\begin{lstlisting}[language=Rust]
pub struct HCSR04<'h, Clock: Timer> {
    trigger: Output<'h>,
    echo: Input<'h>,
    clock: Clock,
}
\end{lstlisting}

Ήδη αρκετά κομμάτια στον κώδικα έχουν ιδιαίτερο ενδιαφέρον.

Αρχικά παρατηρούμε ότι δεν χρησιμοποιούμε συγκεκριμένο τύπο για
GPIO pins. Χρησιμοποιώντας \verb|Input| και \verb|Output| όχι μόνο
μοντελοποιούμε καλύτερα τον αισθητήρα αλλά αποφεύγουμε μελλοντικά
runtime errors αν αφήναμε τον χρήστη να προσθέσει ότι pins ήθελε με
τις δικές του ρυθμίσεις. Με αυτόν τον τρόπο εκφράζουμε "ο αισθητήρας καταναλώνει
ένα output και ένα input pin".

Με την ίδια λογική ορίζουμε μια generic παράμετρο Clock η οποία περιορίζεται
στους τύπους που υλοποιούν το \verb|Timer| trait. Αφήνουμε λοιπόν στον
χρήστη να επιλέξει το ρολόι που θα χρησιμοποιήσει εφόσον οι περισσότεροι
μικροελεγκτές έχουν πολλά και ανάλογα με τους σκοπούς τα αποτελέσματα
μπορούν να διαφέρουν. Για τον ESP, αν μας ενδιαφέρει πάρα πολύ η εγκυρότητα των τιμών
του χρονομετρητή τότε συνήθως θέλουμε το system timer αλλιώς μπορούμε να επιλέξουμε
ένα από τα Timer Groups που προσφέρει ο ESP ή ένα software timer της embassy το οποίο
προφανώς βασίζεται σε ένα από τα παραπάνω.

Το συντακτικό \verb|<'h>| έχει να κάνει με τον χρόνο ζωής (lifetime)
των πεδίων στο εσωτερικό του struct. Η πλήρης ανάλυση των lifetimes
είναι εκτός των πλαισίων της εργασίας αλλά ουσιαστικά αυτό που
εκφράζουν είναι για πόσο χρόνο κατά την εκτέλεση του προγράμματος ζει
ένα σύμβολο. Στην συγκεκριμένη περίπτωση όλα ζουν για το ίδιο χρόνικό
διάστημα, μέχρι ο HCSR04 να βγει από το scope.

Όσο αναφορά τις μεθόδους του struct τις γράφουμε μέσα σε ένα \verb|impl|
block που διαθέτει τις κατάλληλες μεταβλητές. Τα \verb|impl| blocks εκφράζουν
το πιο μέθοδοι ορίζονται για έναν τύπο $T$.

\begin{lstlisting}[language=Rust]
  impl<'h,Clock: Timer> HCSR04<'h, Clock>
  {
    //Methods go here
  }
\end{lstlisting}

Οι μεταβλητές αυτές είναι όσο πιο γενικές δυνατόν γίνεται. Θα μπορούσαμε
κάλιστα να ορίσουμε ένα πιο περιορισμένο implementation. Για παράδειγμα να
ορίσουμε μεθόδους μόνο για τα HCSR04 που έχουν για χρονομετρητή ένα \verb|Alarm|:

\begin{lstlisting}[language=Rust]
  impl<'h> HCSR04<'h, Alarm>
  {
    //Methods go here
  }
\end{lstlisting}

Χρειαζόμαστε προφανώς μια συνάρτηση για να δημιουργήσουμε
το αντικείμενο που αντιπροσωπεύει τον αισθητήρα.

\begin{lstlisting}[language=Rust]
//inside the impl block
pub fn new<TRIG: PeripheralOutput + OutputPin, ECHO: PeripheralInput + InputPin>(
trigger: impl Peripheral<P = TRIG> + 'h,
echo: impl Peripheral<P = ECHO> + 'h,
timer: Clock,
) -> Self {
Self {
    trigger: Output::new(trigger, Level::Low),
    echo: Input::new(echo, Pull::None),
    clock: timer,
}
}
\end{lstlisting}

Η συνάρτηση έχει δύο παραμέτρους που είναι generic και πρέπει
να κάνουν implement το Peripheral trait (διαφορετικό από το Peripheral struct).

\begin{enumerate}
\item Η trigger είναι μια παράμετρος που κάνει implement το Peripheral
  και είναι PeripheralOutput + OutputPin. Δηλαδή αναγκάζουμε την συνάρτηση
  να δεχτεί GPIO Pins που λειτουργούν σαν outputs. Έστω ότι για κάποιο
  λόγο προσπαθήσουμε να καλέσουμε την συνάρτηση με trigger = VCC Pin
  το πρόγραμμα δεν θα κάνει compile επειδή το VCC Pin δεν κάνει implement
  το OutputPin.
\item Η echo είναι μια παράμετρος που κάνει implement το Peripheral
  και είναι PeripheralInput + InputPin. Δηλαδή αναγκάζουμε την συνάρτηση
  να δεχτεί GPIO Pins που λειτουργούν σαν inputs. Έστω ότι για κάποιο
  λόγο προσπαθήσουμε να καλέσουμε την συνάρτηση με echo = VCC Pin
  το πρόγραμμα δεν θα κάνει compile επειδή το VCC Pin δεν κάνει implement
  το InputPin.
\end{enumerate}

Η πιο ενδιαφέρουσα μέθοδος είναι αυτή που υπολογίζει την απόσταση:

\begin{lstlisting}[language=Rust]
//inside the impl block
pub fn measure_distance(&mut self) -> f64 {
    let delay = Delay::new();

    self.clock.start();
    //Send ultrasonic wave
    self.trigger.set_high();
    delay.delay_micros(10);
    self.trigger.set_low();
    //Ultra sonic wave is sent

    //Wait until echo bounces back
    while self.echo.is_low() {
        //DO NOT DEOPTIMIZE THE LOOP AWAY
        unsafe { asm!("nop") };
    }

    //Start timing start = CURRENT_TIME
    let start = self.clock.now();

    //Wait until echo pin returns to low
    while self.echo.is_high() {
        //DO NOT DEOPTIMIZE THE LOOP AWAY
        unsafe { asm!("nop") };
    }

    //End timing end = CURRENT_TIME > start
    let end = self.clock.now();

    let pulse_duration = (end - start).to_micros();

    //return the distance
    pulse_duration as f64 / 58_f64
}
\end{lstlisting}

Η συνάρτηση είναι blocking και υλοποιείται πλήρως συγχρονισμένα. 
Το πρώτο βήμα είναι να δημιουργήσουμε μια καθυστέρηση με υψηλή ακρίβεια μέσω της εντολής:

\begin{lstlisting}[language=Rust]
let delay = Delay::new();
\end{lstlisting}

Έπειτα ξεκινάει το χρονόμετρο ώστε να μπορούμε να μετρήσουμε τη διάρκεια του παλμού echo:

\begin{lstlisting}[language=Rust]
self.clock.start();
\end{lstlisting}

Στέλνουμε ένα κύμα υπερήχων στον αισθητήρα σηκώνοντας το trigger pin σε κατάσταση \verb|HIGH| για $10\mu s$:

\begin{lstlisting}[language=Rust]
self.trigger.set_high();
delay.delay_micros(10);
self.trigger.set_low();
\end{lstlisting}

Μετά, περιμένουμε να επιστρέψει η αντανάκλαση του κύματος. Όσο το
\verb|echo| pin παραμένει \verb|LOW| (δηλαδή δεν έχει φτάσει ακόμα η
αντανάκλαση), παραμένουμε σε μια ενεργή αναμονή (busy wait). Το
\verb|unsafe { asm!("nop") }| διασφαλίζει ότι η αναμονή δεν θα
βελτιστοποιηθεί εκτός από τον compiler.

\begin{lstlisting}[language=Rust]
while self.echo.is_low() {
    unsafe { asm!("nop") };
}
\end{lstlisting}

Όταν ανιχνευτεί η μετάβαση σε \verb|HIGH|, ξεκινάει η χρονομέτρηση του παλμού:

\begin{lstlisting}[language=Rust]
let start = self.clock.now();
\end{lstlisting}

Ακολουθεί μια δεύτερη αναμονή έως ότου το pin \verb|echo| επιστρέψει σε κατάσταση \verb|LOW|, δηλαδή όταν τελειώσει ο παλμός:

\begin{lstlisting}[language=Rust]
while self.echo.is_high() {
    unsafe { asm!("nop") };
}
let end = self.clock.now();
\end{lstlisting}

Η διαφορά των δύο χρονικών στιγμών δίνει την διάρκεια του παλμού σε μικροδευτερόλεπτα:

\begin{lstlisting}[language=Rust]
let pulse_duration = (end - start).to_micros();
\end{lstlisting}

Τέλος, επιστρέφουμε την υπολογισμένη απόσταση χρησιμοποιώντας τον τύπο:

\[
\text{απόσταση (σε εκατοστά)} = \frac{\text{χρόνος (μικροδευτερόλεπτα)}}{58}
\]

\begin{lstlisting}[language=Rust]
pulse_duration as f64 / 58_f64
\end{lstlisting}

Ο συγκεκριμένος τύπος βασίζεται στην ταχύτητα του ήχου στον αέρα και
λαμβάνει υπόψη το γεγονός ότι ο ήχος διανύει την απόσταση δύο φορές
(προς και πίσω). Ο αριθμός $58$ προκύπτει από την αναγωγή της εξίσωσης
$\Delta x = \frac{v \cdot t}{2}$ σε μορφή που επιστρέφει απόσταση σε
εκατοστά.

Αξιοσημείωτο είναι επίσης το γεγονός ότι δεν χρησιμοποιείται καμία
\verb|await| ή ασύγχρονη λογική. Αυτό γίνεται για να εξασφαλιστεί
ακρίβεια και συνέπεια στην μέτρηση χρόνου, καθώς οποιαδήποτε διακοπή
θα αλλοίωνε το αποτέλεσμα. Η προσέγγιση αυτή είναι ιδανική για το
portability του κώδικα καθώς μπορεί να τρέξει σε περιβάλλοντα τα οποία
δεν έχουν executors και κατα συνέπεια async engines. Το να
χρησιμοποιήσουμε μια μη-async συνάρτηση σε έναν executor είναι πολύ
απλό οπότε συνήθως αυτός είναι ο τρόπος που προτείνεται το design.

\subsubsection{MQTT Driver}

Για την χρήση MQTT χρησιμοποιούμε την βιβλιοθήκη \verb|rust-mqtt|
και την \verb|serde| για να κάνουμε serialize τα structs σε json.

Η βιβλιοθήκη \verb|rust-mqtt| έχει αρκετά προβλήματα και αυτός είναι
ο λόγος που αποφασίστηκε να δημιουργήσουμε ένα wrapper για το πείραμα.

Μας κάνει export ένα mqtt client struct:

\begin{lstlisting}[language=Rust]
pub struct MqttClient<'a, T, const MAX_PROPERTIES: usize, R: RngCore>
where
    T: Read + Write,
{
    raw: RawMqttClient<'a, T, MAX_PROPERTIES, R>,
}
\end{lstlisting}

Το οποίο φαίνεται προφανώς να είναι ένα wrapper για το πραγματικό client:

\begin{lstlisting}[language=Rust]
pub struct RawMqttClient<'a, T, const MAX_PROPERTIES: usize, R: RngCore>
where
T: Read + Write,
{
connection: Option<NetworkConnection<T>>,
buffer: &'a mut [u8],
buffer_len: usize,
recv_buffer: &'a mut [u8],
recv_buffer_len: usize,
config: ClientConfig<'a, MAX_PROPERTIES, R>,
}
\end{lstlisting}

Μπορεί κατά πρώτη όψη να μην είναι προφανές το πρόβλημα, αλλά
ελέγχοντας πλήρως τον κώδικα δεν γίνεται καθόλου διατήρηση (ούτε καν
αρχικοποίηση) των εσωτερικών buffer του \verb|RawMqttClient| σε κανένα
από τα παραπάνω structs. Ο χρήστης είναι υπεύθυνος να δημιουργήσει τα buffers,
το socket που ακούει και στέλνει και να τα περάσει στο client. Αν χρησιμοποιούσαμε
την βιβλιοθήκη κατευθείαν ο κώδικας θα ήταν δυσανάγνωστος στην καλύτερη περίπτωση οπότε
προσθέτουμε ένα παραπάνω επίπεδο αφαίρεσης:

\begin{lstlisting}[language=Rust]
pub struct MqttDriver<'a> {
client: MqttClient<'a, &'a mut TcpSocket<'static>, 5, CountingRng>,
}

impl<'a> MqttDriver<'a>
{
}
\end{lstlisting}

Ο σκοπός είναι να κρύψουμε την διαχείριση του TCP socket από
τον χρήστη οπότε αναγκαστηκα δημιουργούμε ένα global socket
μόλις ο χρήστης κάνει import την βιβλιοθήκη. Παράλληλα
αρχικοποιούμε τα buffers για το TCP socket και το MQTT client.

\begin{lstlisting}[language=Rust]
//Global variables
//MQTT BUFFERS
static mut RECV_BUFFER: [u8; 4096] = [0; 4096];
static mut WRITE_BUFFER: [u8; 4096] = [0; 4096];
//TCP SOCKET BUFFERS
static mut RX_BUFFER: [u8; 4096] = [0; 4096];
static mut TX_BUFFER: [u8; 4096] = [0; 4096];
//SOCKET STATICCELL
static mut SOCKET: StaticCell<TcpSocket> = StaticCell::new();
\end{lstlisting}

Δεν μπορούμε να δημιουργήσουμε κατευθείαν ένα TcpSocket καθώς η
μέθοδος \verb|TcpSocket::new| έχει το signature
\verb|new(Stack,&mut [u8],&mut [u8])| όπου stack είναι το network
interface που δημιουργείτε αργότερα κατά την εκτέλεση του
προγράμματος. Παράλληλα δεν μπορούμε να κάνουμε allocate μνήμη στο
Heap επειδή δεν ξέρουμε αν θα υπάρχει απαραίτητα. Το \verb|StaticCell|
είναι ένας έξυπνος δείκτης που δείχνει στην στατική μνήμη του
προγράμματος. Γνωρίζουμε το (μέγιστο) μέγεθος του TcpSocket οπότε
μπορούμε να δεσμεύσουμε μνήμη ίση με αυτό το μέγεθος και να δείξουμε
σε αυτήν χωρίς να την αρχικοποιήσουμε. Αργότερα όταν παραχθεί ένα
Stack καλούμε μια μέθοδο \verb|init| για να θέσουμε τον δείχτη
ίσο με το αποτέλεσμα του \verb|TcpSocket::new|.

Ο χρήστης επίσης θα πρέπει να προσθέσει πολλές δικές του ρυθμίσεις
οι οποίες μερικές μπορούν και μερικές δεν μπορούν να αποφευχθούν:

\begin{enumerate}
\item Για πόσο η σύνδεση να είναι Keep-Alive (Optional).
\item Το μέγιστο μέγεθος των πακέτων (πρέπει να είναι μικρότερο από το μέγεθος των buffers προφανώς) (Optional).
\item Την IP του εξωτερικού broker (Necessary).
\item Το ID του ως client (Necessary).
\item Το Username του ως client (Necessary).
\item Το Password του ως client (Necessary).
\end{enumerate}

Αυτό φαίνεται πολύ καλό use-case για το builder pattern:

\begin{lstlisting}[language=Rust]
//inside MqttDriver impl block
pub fn new() -> MqttDriverBuilder<'a> {
MqttDriverBuilder {
    id: "id",
    username: "username",
    pwd: "password",
    keep_alive: None,
    packet_size: None,
    broker: (IpAddress::v4(0, 0, 0, 0), 1883),
}
}
\end{lstlisting}

Όπου το \verb|MqttDriverBuilder| είναι:

\begin{lstlisting}[language=Rust]
 pub struct MqttDriverBuilder<'a> {
    id: &'a str,
    username: &'a str,
    pwd: &'a str,
    keep_alive: Option<u16>,
    packet_size: Option<u32>,
    broker: (IpAddress, u16),
}

impl<'a> MqttDriverBuilder<'a> {
    pub fn with_id(mut self, id: &'a str) -> Self {
        self.id = id;
        self
    }

    pub fn with_username(mut self, username: &'a str) -> Self {
        self.username = username;
        self
    }

    pub fn with_password(mut self, pwd: &'a str) -> Self {
        self.pwd = pwd;
        self
    }

    pub fn with_keep_alive(mut self, t: u16) -> Self {
        self.keep_alive = Some(t);
        self
    }

    pub fn with_packet_size(mut self, s: u32) -> Self {
        self.packet_size = Some(s);
        self
    }

    pub fn with_broker_ip(mut self, ip: IpAddress, port: u16) -> Self {
        self.broker = (ip, port);
        self
    }

    pub async fn build(self, stack: Stack<'static>) -> Result<MqttDriver<'a>, ConnectError> {
        let mut conf: ClientConfig<'_, 5, CountingRng> =
            ClientConfig::new(MqttVersion::MQTTv5, CountingRng(20000));
        conf.add_client_id(self.id);
        conf.add_username(self.username);
        conf.add_password(self.pwd);
        conf.max_packet_size = self.packet_size.unwrap_or(1024);
        conf.keep_alive = self.keep_alive.unwrap_or(1000);

        let socket = unsafe { SOCKET.init(TcpSocket::new(stack, &mut RX_BUFFER, &mut TX_BUFFER)) };

        let status = socket.connect(self.broker).await;

        match status {
            Ok(_) => Ok(MqttDriver {
                client: MqttClient::new(
                    socket,
                    unsafe { &mut WRITE_BUFFER },
                    4096,
                    unsafe { &mut RECV_BUFFER },
                    4096,
                    conf,
                ),
            }),
            Err(err) => Err(err),
        }
    }
} 
\end{lstlisting}

Η μέθοδος \verb|build| είναι η τελευταία μέθοδος της αλυσίδας που
πρέπει να καλεστεί. Καταναλώνει τον builder, αρχικοποιεί τον δείκτη
στατικής μνήμης, δηλαδή το TcpSocket, και προσπαθεί να δημιουργήσει
μια TCP σύνδεση με τον broker. Αν αυτή η σύνδεση είναι επιτυχής
επιστρέφεται το MqttDriver αν όχι επιστρέφετε ένα σφάλμα και ο χρήστης
είναι υπεύθυνος για το πως θα διαχειριστεί.

Τώρα απομένουν οι βασικές ενέργειες που πρέπει να κάνει το MqttDriver

\begin{enumerate}
\item Να μπορεί να καθιερώσει μέσο της TCP σύνδεσης μια MQTT σύνδεση.
\item Να μπορεί να δημοσιεύσει ένα μήνυμα σε ένα topic στον broker.
\end{enumerate}

Αυτές οι ενέργειες μπορούν να αντιπροσωπευτούν σε δύο ξεχωριστές μεθόδους

\begin{lstlisting}[language=Rust]
//in MqttDriver impl block
pub async fn connect(&mut self) -> Result<(), ReasonCode> {
    self.client.connect_to_broker().await
}

pub async fn publish(&mut self, message: Message) -> Result<(), ReasonCode> {
    let mut json = [0u8; 248];
    let _len = serde_json_core::to_slice(&message, &mut json).unwrap();
    let a = core::str::from_utf8(&json).unwrap();
    self.client
        .send_message("data", a.as_bytes(), QualityOfService::QoS0, false)
        .await
}
\end{lstlisting}

Το \verb|Message| είναι απλώς ένα struct που μέσο macros κάνει implement
δύο traits του Serde:

\begin{enumerate}
\item Το Serialize που επιτρέπει την μετατροπή του struct Message σε ένα
  String της μορφής JSON.
\item Το Deserialize που επιτρέπει την μετατροπή ενός String της
  μορφής JSON σε ένα struct Message.
\end{enumerate}

και φαίνεται έτσι:

\begin{lstlisting}[language=Rust]
#[derive(Serialize, Deserialize)]
pub struct Message {
distance: f64,
cpu_usage: f64,
}

impl Message {
pub fn new(distance: f64, cpu_usage: f64) -> Self {
    Self {
        distance,
        cpu_usage,
    }
}
}
\end{lstlisting}

Τα traits μας δίνουν πρόσβαση στην συνάρτηση \verb|serde_json_core::to_slice(T,&mut [u8])|
τα οποία δημιουργούν το JSON σε bytes με βάση την τιμή του Message.

\subsubsection{Main Program}

Πλέον έχουμε τις κατάλληλες βιβλιοθήκες για να δημιουργήσουμε το
τελικό project. Για το κύριο πρόγραμμα μπορούμε να ξεκινήσουμε τις
προφανής global μεταβλητές που χρειάζονται

\begin{lstlisting}[language=Rust]
//Resources
static EXECUTOR: StaticCell<Executor> = StaticCell::new();
static WIFI_CONTROLLER: StaticCell<EspWifiController> = StaticCell::new();
static STA_STACK_RESOURCES: StaticCell<StackResources<3>> = StaticCell::new();
//Signals
static RUN_MQTT: Signal<CriticalSectionRawMutex, bool> = Signal::new();
static DISTANCE_SIGNAL: Signal<CriticalSectionRawMutex, f64> = Signal::new();
static CPU_SIGNAL: Signal<CriticalSectionRawMutex, f64> = Signal::new();
//WiFi Credentials
const SSID: &str = "some_ssid";
const PWD: &str = "some_pwd";
\end{lstlisting}

Τα StaticCells είναι απλώς δείκτες όπως έχουμε εξηγήσει για έναν Executor, WifiController και
ένα Stack για ένα WiFi station, θα αρχικοποιηθούν αργότερα. Τα Signals είναι global μεταβλητές
που σταματάνε την ροή του task αλλά δεν είναι blocking στον executor επιτρέποντας σε άλλα tasks
να τρέξουνε, εσωτερικά λειτουργούν σαν mutexes.

Χρειαζόμαστε επίσης κάποιου είδους allocator για ένα μικρό heap που είναι απαραίτητο για το WiFi και μόνο:

\begin{lstlisting}[language=Rust]
extern crate alloc;
const HEAP_SIZE: usize = 32 * 1024;
static mut HEAP: MaybeUninit<[u8; HEAP_SIZE]> = MaybeUninit::uninit();
\end{lstlisting}

Πρέπει να κάνουμε annotate την κύρια συνάρτηση ως main για να ξέρει
ο linker ότι αυτό είναι το entry point του προγράμματος.

\begin{lstlisting}[language=Rust]
#[main]
fn main() -> ! {
  //program code here
}
\end{lstlisting}

Έχουμε ήδη αναφέρει το πως θέλουμε να λειτουργεί ο εκτελεστής έτσι
ώστε να μπορέσουμε να μετρήσουμε το CPU usage. Πριν αρχίσει το
πρόγραμμα να τρέχει όμως πρέπει να κάνουμε set up των drivers και
να αρχικοποιήσουμε σημαντικές μεταβλητές όπως το heap:

\begin{lstlisting}[language=Rust]
//Inside main
//Run in max cpu freq
let config = esp_hal::Config::default().with_cpu_clock(CpuClock::max());
//get peripherals
let peripherals = esp_hal::init(config);
//enable logging
esp_println::logger::init_logger_from_env();
//Initialize the heap
unsafe {
    esp_alloc::HEAP.add_region(esp_alloc::HeapRegion::new(
        HEAP.as_mut_ptr() as *mut u8,
        HEAP_SIZE,
        esp_alloc::MemoryCapability::Internal.into(),
    ));
  }

//Embassy Peripherals
let timg1 = TimerGroup::new(peripherals.TIMG1);
let timg0 = TimerGroup::new(peripherals.TIMG0);
let mut rng = Rng::new(peripherals.RNG);
let wifi = peripherals.WIFI;

//HCSR04 Peripherals
let timsys = SystemTimer::new(peripherals.SYSTIMER);
let trigger = peripherals.GPIO11;
let echo = peripherals.GPIO10;
\end{lstlisting}

Η αρχικοποίηση του WiFi φαίνεται λίγο δυσνόητη αλλά
στην πραγματικότητα είναι απλώς υπερβολικά πολύ verbose

\begin{enumerate}
\item Πρώτα αρχικοποιούμε τον WifiController με ένα
  Timer, ένα Rng και ένα Radio Clock.
\item Έπειτα θέτουμε το mode του controller ως station, αυτό
  μας επιστρέφει ένα wifi client driver τύπου \verb|WifiDevice|
  και τον καινούργιο controller με το mode που θέσαμε.
\item Μετά χρειαζόμαστε ένα configuration για να πούμε στο client το
πως θα συνδεθεί σε ένα Router (Username,Password,Encryption
Protocol)και το πως θα πάρει IP. Εμείς θα χρησιμοποιήσουμε το default
DHCP Client configuration.
\item Τέλος δημιουργούμε το TCP/IP stack με ένα random seed,τα
  resources που έχουμε στην στατική μνήμη και τα configurations που
  δημιουργήσαμε. 
\end{enumerate}

\begin{lstlisting}[language=Rust]
    let wifi_controller =
        &*WIFI_CONTROLLER.init(esp_wifi::init(timg0.timer0, rng, peripherals.RADIO_CLK).unwrap());

    let sta_resources = STA_STACK_RESOURCES.init(StackResources::new());
    let (wifi_sta, mut controller) =
        esp_wifi::wifi::new_with_mode(wifi_controller, wifi, WifiStaDevice).unwrap();

    let sta_config = embassy_net::Config::dhcpv4(Default::default());
    let rng_seed = (rng.random() as u64) << 32 | rng.random() as u64;
    let conf = Configuration::Client(ClientConfiguration {
        ssid: String::<32>::from_str(SSID).unwrap(),
        password: String::<64>::from_str(PWD).unwrap(),
        auth_method: AuthMethod::WPA2Personal,
        ..Default::default()
    });
    controller.set_configuration(&conf).unwrap();
    let (sta_stack, sta_runner) = embassy_net::new(wifi_sta, sta_config, sta_resources, rng_seed);
\end{lstlisting}

Στο τελευταίο βήμα μας επιστρέφετε ένας τύπος \verb|Runner| και το \verb|Stack|. Πλέον ήμαστε έτοιμοι
να αρχικοποιήσουμε το embassy και τον HCSR04 και τον executor:

\begin{lstlisting}[language=Rust]
//Embassy Initialization
esp_hal_embassy::init(timg1.timer0);
//HCSR04
let hcsr04 = HCSR04::new(trigger, echo, timsys.alarm0);
//MQTT Driver

//Executor
let executor = EXECUTOR.init(Executor::new(3 as *mut ()));
let spawner = executor.spawner();
spawner.must_spawn(sta_connection(sta_runner));
spawner.must_spawn(wifi_connection(controller));
spawner.must_spawn(mqtt(sta_stack));
spawner.must_spawn(measure_distance(hcsr04));
//Spawn tasks
let mut acc = 0.0;
let start_time = Instant::now();
loop {
    let start_pol = Instant::now();
    unsafe { executor.poll() };
    let end_pol = Instant::now();
    let elapsed_pol = end_pol.duration_since(start_pol).as_ticks() as f64;
    acc += elapsed_pol;
    let total_elapsed = start_time.elapsed().as_ticks() as f64;
    CPU_SIGNAL.signal(acc * 100_f64 / total_elapsed);
}
\end{lstlisting}

Προφανώς δεν έχουμε γράψει ακόμα τα tasks που κάνουν spawn:

\begin{enumerate}
  \item Το \verb|sta_connection| απλώς θα καλεί την μέθοδο του runner \verb|runner.run()|.
  \item Το \verb|wifi_connection| θα περιμένει μέχρι το station είναι έτοιμο. Μόλις είναι θα ξυπνήσει το mqtt.
  \item Το \verb|mqtt| κοιμάται μέχρι να ξυπνήσει από το \verb|wifi_connection| μέσο ενώς \verb|RUN_MQTT| signal.
    Μόλις ξυπνήσει ελέγχει αν έχει λάβει ipv4 και μπορεί να επικοινωνήσει με τον έξω κόσμο.
    Μόλις λάβει ηλεκτρονική διεύθυνση συνδέεται στον broker και περιμένει για τα signals CPU και DISTANCE τα οποία μόλις
    λάβει τα στέλνει στον broker.
  \item Το \verb|measure_distance| μετράει την απόσταση που υπολογίζει ο αισθητήρας HCSR04 μέσο του driver και στέλνει μέσο
    του signal την τιμή αυτής της απόστασης στο mqtt.    
\end{enumerate}

Ένα task είναι απλώς μια συνάρτηση που είναι annotated με ένα macro \verb|#[task]|:

\begin{lstlisting}[language=Rust]
#[task]
async fn sta_connection(mut runner: Runner<'static, WifiDevice<'static, WifiStaDevice>>) {
runner.run().await;
}

#[task]
async fn wifi_connection(mut controller: WifiController<'static>) {

controller.start_async().await.unwrap();
match controller.connect_async().await {
    Ok(_) => (),
    Err(err) => info!("Error connecting {:#?}", err),
}

loop {
    match esp_wifi::wifi::sta_state() {
        WifiState::StaConnected => {
            //Signal status that station has connected.
            RUN_MQTT.signal(true);
            break;
        }
        e => info!("Other {:#?}", e),
    }
    TaskTimer::after_millis(500).await;
}
}

#[task]
async fn measure_distance(mut sensor: HCSR04<'static, Alarm>) {
loop {
    DISTANCE_SIGNAL.signal(sensor.measure_distance());
    TaskTimer::after_millis(250).await;
}
}

#[task]
async fn mqtt(sta_stack: Stack<'static>) {
//Wait for signal that WiFi_Sta has connected to a network
let _run = RUN_MQTT.wait().await;

//Wait to get IP.
loop {
    if let Some(config) = sta_stack.config_v4() {
        break;
    }
    TaskTimer::after_millis(500).await;
}
//TCP Connect to Broker

let mqtt_driver = MqttDriver::new()
    .with_broker_ip(IpAddress::v4(77, 237, 242, 215), 1883)
    .with_id("esp32c6Client")
    .with_password("chris")
    .with_username("chris")
    .with_packet_size(1024)
    .with_keep_alive(300)
    .build(sta_stack)
    .await;

match mqtt_driver {
    Ok(mut mqtt) => {
        //MQTT connect to broker
        let status = mqtt.connect().await;
        match status {
            Ok(_) => loop {
                let current_distance = DISTANCE_SIGNAL.wait().await;
                let cpu_usage = CPU_SIGNAL.wait().await;
                let msg = Message::new(current_distance, cpu_usage);

                if let Err(err) = mqtt.publish(msg).await {
                    info!("Received reason code {:#?}", err);
                }
            },
            Err(err) => info!("Mqtt Error: {:#?}", err),
        }
    }
    Err(err) => info!("Tcp Error: {:#?}", err),
}
}
\end{lstlisting}


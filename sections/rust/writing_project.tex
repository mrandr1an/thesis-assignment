\subsection{Γράφοντας το Project}

Θα δούμε πως η υλοποίηση του Project με βάση το specification που
έχουμε θέσει είναι ιδιαίτερα απλή. Παρόλα αυτά δεν προσφέρονται βιβλιοθήκες
για τον HC-SR04 όπως στην περίπτωση του ESP-IDF. Παράλληλα το API της
βιβλιοθήκης του MQTT είναι πολύ δύσκολα διαχειρήσημο οπότε θα
χρειαστεί κάποιου είδους βελτίωση χωρίς απαραίτητα να γράψουμε το
πρωτόκολλο από την αρχή (το οποίο όμως φαίνεται να είναι η ιδανική
λύση). Με βάση την γνώση που έχουμε δώσει στην εισαγωγή αυτού του
κεφαλαίου όμως γίνεται ξεκάθαρο πως αυτό δεν αποτελεί πρόβλημα καθώς η
εκφραστηκότητα της Rust κάνει το γράψιμο κώδικα να φαίνεται σαν πολύ
οργανική διαδικασία.

Μια επιλογή για να ξεκινήσουμε είναι να δημιουργήσουμε ένα Cargo.toml και
να θέσουμε τις βιβλιοθήκες που θέλουμε να μεταγλωττίσουμε μαζί με το Project.
Όμως υπάρχει μια καλύτερη λύση, το \verb|esp-generate --chip esp32c6 my_project|
δημιουργεί τα απαραίτητα αρχεία με τις βιβλιοθήκες που είναι απαραίτητες ανάλογα με
τις ρυθμίσεις μας. Προφανώς αν θέλουμε προσθέτουμε κιάλλες μπορούμε είτε από ένα τερματικό
\verb|cargo add library| είτε με το συντακτικό TOML που έχουμε δείξει παραπάνω.
Το τελικό directory έχει της εξής δομή:

\begin{figure}[htbp]
  \centering
\begin{forest}
for tree={
  font=\ttfamily,
  grow'=0,
  child anchor=west,
  parent anchor=south,
  anchor=west,
  calign=first,
  edge path={
    \noexpand\path [draw, \forestoption{edge}]
      (!u.south west) +(5pt,0) |- (.child anchor)\forestoption{edge label};
  },
  before typesetting nodes={
    if n=1
      {insert before={[,phantom]}}
      {}
  },
  fit=band,
  before computing xy={l=15pt},
}
[.
  [build.rs]
  [.cargo
    [config.toml]
  ]
  [Cargo.toml]
  [.gitignore]
  [rust-toolchain.toml]
  [src
    [bin
      [main.rs]
    ]
    [lib.rs]
  ]
]
\end{forest}
\caption{Project folder structure.}
\label{fig:project_structure}
\end{figure}

Το project θα αποτελείτε από δύο δικές μας βιβλιοθήκες που θα είναι υπεύθυνες για:

\begin{enumerate}
  \item Την λειτουργία του αισθητήρα απόστασης.
  \item Την λειτουργεία του πρωτόκολλου MQTT.
\end{enumerate}

\subsubsection{HC-SR04 Driver}
Hello 
\subsubsection{MQTT Driver}
Goodbye
\subsubsection{Main Program}
Hellogoodbye
